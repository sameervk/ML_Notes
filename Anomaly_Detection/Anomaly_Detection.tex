\documentclass[a4paper, 12pt]{report}

\usepackage{hyperref}
\usepackage{titlesec}

\usepackage{lipsum} %for random text: use as \lipsum[num1 or num1-num2]

\usepackage{color} %can change the background of the text also


\titleformat{\chapter}[display]
  {\normalfont\bfseries}{}{0pt}{\Huge}


\begin{document}
\title{Notes on Anomaly Detection}
\author{Sameer Kesava PhD}
\date{} % or {Month year}
\maketitle

\pagenumbering{roman}
\tableofcontents
\newpage

\pagenumbering{arabic}

%\setcounter{section}{0}



%-----------------------------------
\chapter{Unsupervised Learning Algorithms}
For more information, please see \textbf{Deep Learning for Anomaly Detection: A Survey}; http://arxiv.org/abs/1901.03407. 
%%----------------
\section{MeanShift Clustering}
\begin{itemize}
\item Distance-based clustering
\item Has the ability to detect outliers.
\end{itemize}

%%----------------
\section{DBSCAN}
Density-based spatial clustering of applications with Noise
\begin{itemize}
\item Hyperparameters: min$\_$samples, $\epsilon$ and $\epsilon$-metric.
\item Has the ability to detect outliers.
\item Works well for non-linear data.
\item Affected by the curse of dimensionality.
\end{itemize}


%-----------------------------------
\chapter{Supervised Learning Algorithms}

%%----------------
\section{Univariate Data}
\begin{itemize}
\item Boxplot
\item Grubbs test
\item RANSAC algorithm for linear regression
\item Studentized residuals and leverage points. Easy to plot for univariate data.
\end{itemize}

%%----------------
\section{Multivariate Data}

%%----------------
\section{Random Cut Forest}
From Amazon SageMaker

%%----------------
\section{XGBoost}
Highly popular classifier and regressor. 
\begin{itemize}
\item Gradient boosting method
\item Absolute loss and Huber loss more robust to outliers.
\item Hyperparameters
\begin{enumerate}
\item Max$\_$depth
\item Colsample$\_$bytree
\item Eta
\item train-test split: 60-40/70-30/80-20.
\end{enumerate}
\end{itemize}

%%----------------
\section{Isolation Forest}


%-----------------------------------
\chapter{Improving the Accuracy}
%%----------------
\section{Hyperparameter Tuning}
\begin{itemize}
\item Hyperparameter optimization based on Gaussian Process Regression and Bayesian Optimization
\item keras tuner in keras
\item GridSearchCV or RandomSearchCV in scikit-learn
\item RandomSearch can be used as the baseline against which optimization algorithms can be evaluated.
\end{itemize}

\begin{thebibliography}{999}
\bibitem{pankaj_2015}
	Pankaj Malhotra et al.,
	Long Short Term Memory Networks for Anomaly Detection in Time Series,
	2015.
\end{thebibliography}










\end{document}
