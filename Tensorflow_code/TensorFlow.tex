\documentclass[11pt]{article}

\usepackage[a4paper, left=20mm, right=20mm, top = 20mm, bottom=20mm]{geometry}

\usepackage{hyperref}

\usepackage{titlesec}

\usepackage{amsmath}

\usepackage{mathtools}

\usepackage{multicol}
\setlength{\columnsep}{0.75cm}

\usepackage{lipsum} %for random text: use as \lipsum[num1 or num1-num2]

\usepackage{color} %can change the background of the text also


%\titleformat{\chapter}[display]{\normalfont\bfseries}{}{0pt}{\Huge}

\setlength{\parindent}{0in}

\title{Notes on TensorFlow}
\author{Sameer Kesava PhD}
%\date{\today} % or {Month year}

\begin{document}
\maketitle
\begin{abstract}
\noindent Important points on coding in tensorflow and keras with tensorflow backend. Tensorflow objects rather than Python objects preferrable for defining and training in tensorflow.
\end{abstract}


%\pagenumbering{roman}
%\tableofcontents
%\newpage

\pagenumbering{arabic}

%\setcounter{section}{0}

\begin{multicols*}{2}

\paragraph{version 2.1 \label{version}}
%%---------------------
\section{Estimators}
\begin{itemize}
\item Estimators are like scikit-learn pipelines.
\item For training models for distributed environments.
\end{itemize}

%%---------------------
\section{Data Formats}
\begin{itemize}
\item Use a combination of protobuf and tf.recordIO to speed up training.
\item Inputs must better be converted to float32 (float64 rarely encountered) since certain layers like dropout or batchnorm will throw out a "Value Error" if the inputs are not of floating point type.
\item If the input file is loaded with numbers in the form of string type, then use \textit{ast.literal$\_$eval()} or \textit{json.loads()} to convert to numbers of type float.
\end{itemize}

%%---------------------
\section{Images}
\begin{itemize}
\item tf.image module for formatting images.
\item Resizing can be done using methods such as bilinear, lanczos, etc.
\item tf.keras.preprocessing.image.array$\_$to$\_$img()
\item tf.keras.preprocessing.image. ImageDataGenerator: excellent function for loading images from disk and \textbf{\textit{augmenting}} data (especially useful in the case of small datasets).
\item tf.image.draw$\_$bounding$\_$boxes(): for drawing bounding boxes on images for image segmentation or objection detection.
\end{itemize}

%%---------------------
\section{Python Scopes}
\begin{enumerate}
\item Global: should be explicitly defined.
\item Nonlocal: same as above.
\item local

\end{enumerate}

Functions within function give greater control when also including scopes such as global and nonlocal.

%%---------------------
\section{Array Broadcasting}
\begin{itemize}
\item In numpy and tensorflow, the smaller array is "broadcast" across a larger array so that they have compatible shapes.
\item \textbf{Broadcasting Rule:} In order to broadcast, the size of the trailing axes for both arrays in an operation must either be the same size or one of them must be one.
\end{itemize}

%%---------------------
\section{Debugging in TensorFlow}
tf.debugging

%%--------------------
\section{Logits}
\begin{itemize}
\item The argument "from$\_$logits" that is passed to the loss function should be carefully reviewed. If the last layer has a softmax activation, then from$\_$logits must be set to False.
\item If the y labels are negative and not one-hot encoded, using from$\_$logits = False will lead to wrong normalization.
\item See \textbf{Candidate Sampling} for training data containing large number of classes.
\end{itemize}


%%---------------------
\section{Transfer Learning}

\subsection{Regularization}
Adding regularizers to pre-trained models is not straightforward.
\begin{enumerate}
\item Load the pre-trained model and add regularizers to the desired layers.
\item Save the model to a json or yaml object.
\item Save the model weights.
\item Delete this model and create a new one using the json object.
\item Load the saved weights into this model.
\item Check for the layers where the regularizers were added, should be created now.
\end{enumerate}


%%----------------------
\section{Padding}
\begin{itemize}
\item tf.keras.preprocessing.sequence.pad$\_$sequences
\item Post-padding is recommended to be able to use CuDNN implementation of the layers.
\end{itemize}

%%----------------------
\section{Masking}
\begin{itemize}
\item Masking is required to inform the model that some part of the data has padding.
\item In models using functional or sequential API, the masks are propagated directly.
\item However, in subclassed models, the mask arg. must be explicity passed to the call method. E.g. the compute$\_$mask method of the embedding layer can be passed as an arg. to the call method of the next layer like LSTM.
\end{itemize}

%%----------------------
\section{Embedding}
\begin{itemize}
\item tf.nn.embedding$\_$lookup(): returns the embedded values of the inputs from the embedding tensor.
\end{itemize}

%%----------------------
\section{Tokenizer}
\begin{itemize}
\item For text generation: tf.keras.preprocessing.text. Tokenizer(). Also used in image captioning with visual attention.
\end{itemize}

%%----------------------
\section{Dense Layer}
\begin{itemize}
\item Can take input of more than 2 dimensions.
\end{itemize}

%%----------------------
\section{Gradient Calculation}
\begin{itemize}
  \item Variables not part of the gradient can be tracked using tape.watch().
  \item \begin{enumerate}
    \item tf.clip$\_$by$\_$value; tf.clip$\_$by$\_$norm for gradient clipping to address exploding gradient problem.
  \end{enumerate}
\end{itemize}

%%----------------------
\section{Other Layers}
\begin{enumerate}
\item Lambda layer: wraps arbitrary expressions such as lambda functions or other functions as a Layer object; helps improve portability of the models.
\item Activation layer: Useful when outputs from multiple inputs are concatenated, and for Batch/Layer normalization layers.
\end{enumerate}

%%----------------------
\section{Useful functions}
It is preferrable to use tf objects, methods and functions as it enables serialization of the code possible and easy visualization in TensorBoard.

\begin{enumerate}
\item Check tf.keras.backend for useful functions.
\item tf.map$\_$fn: applies a function to a tensor along an axis.
\item tf.reduce$\_$sum: sum along axes; another important parameter is keepdims.
\item tf.square: squaring the elements.
\item tf.cond(bool, true func, false func): returns only a func. 
	\begin{itemize}
	\item Use lambda expression to return a tensor or other output or the functions must be callable. 
	\item Objects in true func cannot be accessed by false func.
	\item Outputs from both must be of the same type and shape.
	\item Functions can return \textbf{None}.
	\end{itemize}
\item tf.nn.moments: for mean and variance. Important args: axes and keepdims.
\item tf.reduce$\_$any: logical or along axes; keepdim arg.
\item tf.reduce$\_$mean: mean along axes; keepdim arg.
\item tf.equal: compares elements between 2 tensors. return bool.
\item TensorSpec: for passing tensors with general shapes.
\end{enumerate}

%%----------------------
\section{Tensor properties \nameref{version} \autoref{version}}
\begin{itemize}

\item tensor.assign() for assigning a new value or manipulating a tensor variable. For a specific index [m,n], use tensor[m,n].assign(). Direct assignment such as var = var + newvalue will cause issues.
\item Do not directly use layer.weights for examining and manipulating, use the .numpy() or .read$\_$value().
\item All tf.Variable objects that are not part of training must have their trainable attribute set to False.
\end{itemize}

%%----------------------
\section{tf.Graph properties \nameref{version} \autoref{version}}
It is very important to understand the graph-based training in tensorflow. The points below are based on my observations, might not be correct. Need to learn more about this.
\begin{itemize}
\item Once a graph is created, the variables are added to the graph. These variables cannot be reinitialized later but can be manipulated using the assign method.
\end{itemize}

%%----------------------
\section{Keras Layer properties \nameref{version} \autoref{version}}
\begin{itemize}
\item A layers state is given by its weights/variables. Some layers are \textbf{stateless}, i.e. in these layers only data transformation occurs.
\item When subclassing layers, parameters such as name, trainable and dtype can be passed to the init method, from where the trainable argument can be controlled.
\item A layer's call arguments are inputs, training and mask.
\end{itemize}


%%----------------------
\section{Keras Model properties \nameref{version} \autoref{version}}
\begin{itemize}
\item Keras.Model tracks both its variables and its internal layers.
\item A keras model can be used in the same way as a keras layer when building deeper models.
\item All the functions and methods of a model, including the functions corresponding to the inference mode, are evaluated when compiled. This is especially relevant for subclassed models.
\item In order to reset some of the Variables in the model or its layers, the layers have to be passed a \textbf{stateful} argument during initialization like in RNN.
	\begin{enumerate}
	\item Directly assigning model.stateful = True will lead to AttributeError as stateful is a readonly property and hence cannot be directly assigned.
	\item Thus, each layer must be passed a stateful argument. Also, it is not a kwarg for a keras layer.
	\end{enumerate}
	
\end{itemize}



%%----------------------
\section{Keras Loss methods \nameref{version} \autoref{version}}
\begin{enumerate}
\item tf.keras.losses.Loss class can be subclassed to define custom loss functions.
\end{enumerate}


\end{multicols*}
\end{document}