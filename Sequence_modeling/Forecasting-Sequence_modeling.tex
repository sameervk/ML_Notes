\documentclass[a4paper, 12pt]{report}

\usepackage{hyperref}
\usepackage{titlesec}

\usepackage{lipsum} %for random text: use as \lipsum[num1 or num1-num2]

\usepackage{color} %can change the background of the text also


\titleformat{\chapter}[display]
  {\normalfont\bfseries}{}{0pt}{\Huge}


\setlength{\parindent}{0in}


\begin{document}
\title{Notes on Forecasting/Sequence Modeling}
\author{Sameer Kesava PhD}
%\date{\today} % or {Month year}
\maketitle

\pagenumbering{roman}
\tableofcontents
\newpage

\pagenumbering{arabic}

%\setcounter{section}{0}

\chapter{Framework}
A set of questions needs to be clearly answered to understand and frame the time-series forecasting problem.

\begin{enumerate}
\item What are the Inputs and Outputs for the forecast?
\item What are the Endogenous and Exogenous variables?
\item Is the data Structured or unstructured?
\item Is it a Regression or Classification problem?
\item Is it a Univariate or Multivariate problem?
\item Does the problem require a Static or Dynamic model?
\item Is the data Contiguous or Discontiguous?
\end{enumerate}

\chapter{Improving the Accuracy}
\section{Issues}
\label{sec:issues}
\begin{itemize}
\item[-] Long term dependencies is a problem.
\end{itemize}

\subsection{Vanishing Gradient}
LSTM or GRU architecture helps resolve this problem.

\subsection{Exploding Gradient}
\begin{enumerate}
\item Gradient clipping
\item Nonlinear activation functions such as relu6.
\end{enumerate}


%%----------------------
\section{Weights}
Parameter sharing is the main feature of RNNs. The parameters, i.e. same weights, are shared across several time steps.
\label{sec:weights}
\begin{enumerate}
\item Sample or Class weights can be used to improve the prediction accuracy.
\item Alternatively, a user-defined function can be supplied to compute the weights.

\subsection{ML}

\subsection{DL}
\begin{enumerate}
\item A good starting values for the initializers are Glorot-Xavier or Orthogonal for the weights.
\item For the bias, zeros suffice.
\end{enumerate}
\end{enumerate}



\section{Brute-Force}
\label{sec:bruteforce}

Brute-force algorithm is effective for small datasets.\\
Query time grows as O[pn], where p is the number of features/predictors/dimensions and n is no. of data points.

\section{Small Dataset}
\label{sec:smalldatasize}
n: number of samples\\
p: number of features/predictors

\subsection{When p is comparable to n}
\begin{enumerate}

\item Use shrinkage methods such as Lasso or L1 norm.
%\item Check \hyperref[bruteforce]{\ref{bruteforce}}.
\item Check \nameref{sec:bruteforce} \autoref{sec:bruteforce}.

\end{enumerate}

\subsection{p ${\ll}$ n}
\begin{enumerate}
\item Use k-fold CV.
\end{enumerate}

\section{Embedding}
Embedding of characters and words in character/language modeling is a popular way of encoding in sequence modeling.

\section{Statefulness}
\begin{itemize}
\item Stateful = True, i.e. in many-to-many sequence modeling, where there is one-to-one mapping between row i of batch j with row i of batch j+1.
\end{itemize}

\section{Norm of the Gradient}
\begin{itemize}
\item During training, the norm of the gradient clipped can also be dynamically adjusted ({\color{green}from NIPS:Oral Session 4 -Ilyasuts}).
\end{itemize}

\section{Curriculum Learning}
To help stabilize the model's open-loop output.

\section{Teacher Forcing}
Used in \nameref{sec:enc-dec} \autoref{sec:enc-dec} where the target word is passed as the next input to the decoder.

\section{Hyperparameter Tuning}
\begin{itemize}
\item Hyperparameter optimization based on Gaussian Process Regression and Bayesian Optimization
\item keras tuner in keras
\item GridSearchCV or RandomSearchCV in scikit-learn
\item RandomSearch can be used as the baseline against which optimization algorithms can be evaluated.
\end{itemize}

%-------------------------------
\chapter{ML Methods}

\section{ARIMA}
\begin{itemize}
\item Implemented in statsmodel.
\item Cannot model seasonal trends. Hence, differencing should be carried out to remove this.
\end{itemize}

%%----------------
\section{XGBoost}
Highly popular classifier and regressor. 
\begin{itemize}
\item Gradient boosting method
\item Absolute loss and Huber loss more robust to outliers.
\end{itemize}



%-------------------------------
\chapter{DL Methods}
All operations, e.g. loss functions, activation functions, in neural networks must be differentiable.

\begin{itemize}
\item If the sequences are not of the same size, e.g. Sentiment Analysis, either padding or masking can be used.
\end{itemize}

%%----------------
\section{Binary Classification}
\begin{itemize}
\item Should end with a dense layer of 1 unit and a sigmoid activation function.
\item The loss function should be binary cross-entropy.
\end{itemize}

%%--------------------
\section{CNN-LSTM}
\label{sec:cnn-lstm}
Hyrbid model.

\begin{itemize}
\item {\color{red}Invariance is hard-coded in CNNs.(?)}

\end{itemize}

%%--------------------
\section{ConvLSTM}
Hybrid model.
%%--------------------
\section{TreeRNN}

%%--------------------
\section{Bidirectional RNN}
\begin{itemize}
\item Helps RNN learn long range dependencies in the sequences.
\item For language modeling.
\end{itemize}

%%--------------------
\section{Encoder-Decoder}
\label{sec:enc-dec}
\begin{enumerate}
\item Bahdanau et al. - RNN encoder, RNN decoder with Attention, i.e. contextual output from the encoder as input to the decoder.
\item In Neural Machine Translation (NMT), where the above method was applied, the encoder output is trained to capture the semantic information, which is then passed to the decoder.
\end{enumerate}

%-------------------------------
\chapter{Loss Functions}

\section{Sparse Categorical Cross Entropy}
To be used when the labels are not one-hot encoded.

\section{Binary Cross Entropy}
\begin{enumerate}
\item If the labels have only 2 unique values, preferably [0,1] or [-1,1].
\item Use sigmoid activation function for [0,1] and tanh for [-1,1].
\end{enumerate}

\section{KL Divergence}
If each label is a probability distribution, e.g. image segmentation.

\section{Perplexity}
A function used in language modeling: e$^{Loss}$

%-----------------------------
\chapter{Metrics}

\section{BLEU score}
Used in language modeling, e.g. {\color{green}Bahdanau et al. paper; \nameref{sec:enc-dec} \autoref{sec:enc-dec}}.


%-----------------------------
\chapter{Optimizers}
\begin{enumerate}
\item If the training and validation accuracies with epoch jump around, then learning rate must be reduced.
\end{enumerate}
\section{Stochastic Gradient Descent}
\begin{itemize}
\item[-] For training one sample at a time.

\item[-] For online learning.
\end{itemize}

\section{Mini-batch Gradient Descent}
\begin{itemize}
\item[-] For mini-batches typically of the order of 2$^x$ for GPU compatibility.
\item[-] Examples are Adam, Adagrad, RMSProp, etc.
\end{itemize}


\chapter{Overfitting}

\section{DL}
\begin{itemize}
\item L1, L2, L1L2 regularization
\item Dropout
\item Batch Normalization
\item Reduce the complexity of the network especially if the data size is small.
\end{itemize}

\subsection{Dropout}
\begin{itemize}

\item For a RNN layer, there are 3 dropouts: for the input from the previous layer, output of the layer and the recurrent dropout.
\item Dropout functions
\begin{enumerate}
\item Gaussian Dropout
\end{enumerate}
\end{itemize}


\end{document}



% Notes
% 1. Several empty lines are treated as one empty line.
% 2. The main function of an empty line is to start a new paragraph.
% 3. In general, LATEX ignores blank lines and other empty space in the .tex file. Two backslashes (\\) can be used to start a new line.
% 4. If you want to add blank space into your document use the \vspace{...} command.\vspace{12pt} will add space equivalent to the height of a 12pt font.